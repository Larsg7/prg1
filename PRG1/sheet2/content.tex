\section*{2.1}
Reihenfolge: \verb+(4 >> 2)+, \verb+**+, \verb+*+, \verb+/+, \verb+%+, \verb+-+, \verb+<+

\section*{2.2}
\begin{enumerate}
    \item \textbf{True} ist vom Typ \verb+bool+
    \item \textbf{"Hallo Welt"} ist vom Typ \verb+str+
    \item \textbf{'23.5'} ist vom Typ \verb+str+
    \item \textbf{17+4j} ist vom Typ \verb+complex+
    \item \textbf{19.} ist vom Typ \verb+float+
    \item \textbf{655321} ist vom Typ \verb+int+
\end{enumerate}
Bestimmt werden kann der Typ mit der Funktion \verb+type()+. Z.B. gibt \verb+type(True)+ \verb+<class 'bool'>+ zurück.

\section*{2.3}
\begin{enumerate}
    \item \verb+i = 1+ $\rightarrow$ $1.0$
    \item \verb+i = 5+ $\rightarrow$ $2.236067977499978$
    \item \verb+i = 9+ $\rightarrow$ $3.000000001396984$
    \item \verb+i = 12+ $\rightarrow$ $3.464101615137755$
    \item \verb+i = 16+ $\rightarrow$ $4.000000000000051$
\end{enumerate}
Diese Funktion berechnet die Wurzel von 1 iterativ.
Das verwendete Verfahren heißt "Heron-Verfahren" \cite{heron}.
Dabei lautet die Iterationsvorschrift
$$x_{n+1} = \frac{1}{2}(x_n + \frac{i}{x_n})$$
Diese konvergiert zu
$$x = \frac{1}{2}(x + \frac{i}{x}) \rightarrow x^2 = i \rightarrow x = \sqrt{i}$$

\section*{2.4}
{Siehe \verb+Lars_Gröber_2.4.py+.